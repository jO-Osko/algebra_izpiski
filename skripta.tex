\documentclass[a4paper]{article}
%AMDG
\usepackage{amsmath, amsthm, amssymb}
\usepackage{hyperref}
\usepackage[slovene]{babel}
\usepackage[utf8]{inputenc}
\usepackage[T1]{fontenc}
\usepackage{epigraph}
\usepackage{pdftexcmds}
\usepackage{fancyref, nameref}
\usepackage{epigraph}
\usepackage{cleveref}
\usepackage{verbatim}

\parindent=0pt


% \epigraphsize{\small}% Default
\setlength\epigraphwidth{8cm}
\setlength\epigraphrule{0pt}

\usepackage{etoolbox}

\makeatletter
\patchcmd{\epigraph}{\@epitext{#1}}{\itshape\@epitext{#1}}{}{}
\makeatother


\newcounter{environment:definition_counter}

\newenvironment{definition}[1][\unskip]
{\vspace{0.5cm}\refstepcounter{environment:definition_counter}\textbf{Definicija \arabic{environment:definition_counter}: \textbf{#1}}\itshape}
{\bigskip}

\newcounter{environment:theorem_counter}

\newenvironment{theorem}[1][\unskip]
{\refstepcounter{environment:theorem_counter}\textbf{Izrek \arabic{environment:theorem_counter}:\textit{#1}} \qquad}
{\bigskip}

\newcounter{environment:statement_counter}

\newenvironment{statement}[1][\unskip]
{\refstepcounter{environment:statement_counter}\textbf{Trditev \arabic{environment:statement_counter}:\textit{#1}} \qquad}
{\bigskip}

\newcounter{example:example_counter}

\newenvironment{example}
{\textbf{Primer:}\\}
{\setcounter{example:example_counter}{0}}

\newenvironment{example_case}
{\refstepcounter{example:example_counter} \arabic{example:example_counter}.}
{\\}

\newenvironment{remark}
{Opomba:}
{}



\begin{document}
\title{Skripta za algebro2}
\author{Filip Koprivec}
\date{\today}
\maketitle

\epigraph{“If I find in myself desires which nothing in this world can satisfy, the only logical explanation is that I was made for another world.”}{--- \textup{C. S. Lewis}}
\newpage

\tableofcontents

\newpage

\begin{comment}
Start of text
\end{comment}

\section{Osnovne algebrske strukture}
\subsection{Binarne operacije}
\begin{definition}[Binarna Operacija]
\label{def:binary_operation}
(tudi dvočlena operacija) $\circ$ na množici $\mathcal{S}$ je preslikava iz $\mathcal{S} \times \mathcal{S} \  v \ \mathcal{S}$.\\ Torej $\circ : \mathcal{S} \times \mathcal{S} \ \to \ \mathcal{S}$
\end{definition}

\begin{example}
Osnovna zgleda binranih operacij na $\mathbb{Z}$ sta:

\begin{example_case}
Seštevanje: $(n,m) \mapsto n+m$
\end{example_case}
\begin{example_case}
Množenje: $(n,m) \mapsto n \times m$
\end{example_case}

\end{example}

Skalarni produkt v $\mathbb{R}^{2}$ \textbf{ni} binarna operacija.

Vektorski produkt v $\mathbb{R}^{3}$ \textbf{je} binarna operacija.
\\\\
\begin{definition}
\label{def:asociative_operation}
Operacija $\circ$ je \textbf{asociativna}, če ustreza enačbi
\begin{equation}
\label{eq:asociative_law}
\forall x,y,z \in \mathcal{S}.\ (x \circ y) \circ z = x \circ (y \circ z)
\end{equation}

\end{definition}

Enakost \ref{eq:asociative_law} imenujemo \textbf{Zakon o asociativnosti}\\
Operacije, ki jih bomo obravnavali bodo praviloma asociativne.

\begin{definition}
\label{def:comutative_operation}
Elementa $x,y \in \mathcal{S}$ \textbf{komutirata}, če velja 
\begin{equation}
\label{eq:comutative_law}
\forall x,y \in \mathcal{S}. x \circ y = y \circ x
\end{equation}
Enakost \ref{eq:comutative_law} imenujemo \textbf{Zakon o komutativnosti}

\end{definition}
\begin{remark}
Kadar je iz konteksta razvidno, o kateri operaciji govorimo, pogosto namesto "$\circ$ je komutativna rečemo tudi $\mathcal{S}$ je komutativna"
\end{remark}

\begin{example}
\begin{example_case}
Operacija + na $\mathbb{Z}$ je tako asociatitivna in komutativna
\end{example_case}
\begin{example_case}
Operacija * na $\mathbb{Z}$ je tako asociatitivna in komutativna
\end{example_case}
\begin{example_case}
Operacija - na $\mathbb{Z}$ \textbf{ni} niti asociativna niti komutativna
\end{example_case}
\begin{remark}
Na opracijo odštevanja gledamo kot na izpeljano operacijo in ne kot na samostojna operacijo, saj jo vpeljemo preko seštevanja in pojma nasprotnega elementa.
\end{remark}

\begin{example_case}
Naj bo $\mathcal{X}$ poljubna neprazna množica. Z $F(\mathcal{X})$ označimo množico vseh preslikav iz $\mathcal{X}$ v $\mathcal{X}$. Naj bosta $f, g \in \mathcal{X}$, potem je $(f,g) \mapsto f \circ g$ (kompozitum funkcij) binarna operacija na $F(\mathcal{X})$.

\begin{remark}
Operacija je asociativna, in kadar $|\mathcal{X}| \geq 2$ ni komutativna
\end{remark}
\end{example_case}
\end{example}

\begin{definition}
\label{def:identity_element}
Naj bo $\circ$ binarna operacija na na $\mathcal{S}$ in $e \in \mathcal{S}$. $e$ se imenuje \textbf{nevtralni element}, če velja 
\begin{equation}
\label{eq:identity_element}
\forall x \in \mathcal{S}. e \circ x = x \circ e = x
\end{equation}

\end{definition}

\begin{example}
\begin{example_case}
$0$ je nevtralni element za seštevanje na $\mathbb{Z}$.
\end{example_case}
\begin{example_case}
$1$ je nevtralni element za množenje na $\mathbb{Z}$.
\end{example_case}
\begin{example_case}
$id_{x}$ (identična preslikava) je nevtralni element za $F(\mathcal{X})$
\end{example_case}
\end{example}

\begin{remark}
Nevtralni element nima zagotovljenega obstoja (recimo $+$ na $\mathbb{N}$ ali $*$ na sodih celih številih).
\end{remark}

\begin{statement}
\label{st:identity_unique}
Če nevtralni element obstaja je en sam
\begin{proof}
\label{pr:identity_unique}
Naj bosta $f,e \in \mathcal{S}$ nevtralna elementa.
$$e = e \circ f \text{\ \ // Ker je f nevtralni element}$$
$$e \circ f = f \text{\ \ // Ker je e nevtralni element}$$
$$e = f $$
\end{proof}
\end{statement}

\begin{definition}
\label{def:left_identity}
Element $e'$ je \textbf{levi nevtralni element}, če velja:
\begin{equation}
\label{eq:left_identity_element}
\forall x \in \mathcal{S}. e' \circ x = x
\end{equation}
\end{definition}

\begin{definition}
\label{def:right_identity}
Element $e''$ je \textbf{desni nevtralni element}, če velja:
\begin{equation}
\label{eq:right_identity_element}
\forall x \in \mathcal{S}. x \circ e'' = x
\end{equation}
\end{definition}

\begin{remark}
Levih in desnih nevtralnih elementov je lahko več

\begin{example}
\begin{example_case}
$\circ : (x,y) \mapsto y$.

Vsak element je levi nevtralni element
\end{example_case}
\begin{example_case}
$0$ je desni nevtralni element za odštevanje v $\mathbb{Z}$
\end{example_case}
\end{example}
\end{remark}

\begin{statement}
\label{st:identity_left_right_both}
Naj bo za operacijo $\circ$ $e'$ levi nevtralni element, $e''$ pa desni nevtralni element. Tedaj velja $e' = e'' = e$(Sta si levi in desni nevtralni element enaka in je(sta) nevtralni element)
\begin{proof}
\label{pr:identity_left_right_both}
$$e' = e' \circ e'' = e''$$
\end{proof}
\end{statement}

\begin{definition}
\label{def:inner_operation}
Naj bo $\circ $ operacija na $\mathcal{S}$ in naj bo $\mathcal{T} \subseteq \mathcal{S}$. Rečemo, da je $\circ$ \textbf{notranja operacija na $\mathcal{T}$} ali da je množica \textbf{$\mathcal{T}$ zaprta za $\circ$ na $\mathcal{T}$} , če velja
\begin{equation}
\label{eq:inner_operation}
\forall t, t' \in \mathcal{T}. t \circ t' \in \mathcal{T}
\end{equation}

\end{definition}

\begin{example}
Množica $\mathbb{N}$ je zaprta za operaciji $+$ in $*$, ni pa zaprta za operacijo $-$.
\end{example}

\begin{definition}
\label{def:outer_operation}
Preslikavi iz $\mathcal{K} \times \mathcal{S}$ v $\mathcal{S}$ kjer $\mathcal{K} != \mathcal{S}$ rečemo \textbf{Zunanja binarna operacija}
\end{definition}

\begin{example}
\begin{example_case}
Množenje vektorja s skalarjem\\
$(\lambda, \vec{x}) \mapsto \lambda\vec{x}$, kjer je $(K = \mathbb{R}, S = \mathbb{R}^n)$\\
$\lambda (x_1, x_2, \dots , x_n) = (\lambda x_1, \lambda x_2, \dots ,\lambda x_n)$
\end{example_case}
\end{example}

\subsection{Polgrupe in monoidi}

\begin{definition}
\label{def:algebraic_structure}
\textbf{Algebrska struktura} je množica, opremljena z eno ali več operacijami (notranjimi ali zunanjimi), ki im ajo določene lastnosti
\end{definition}

\begin{definition}
\label{def:semigroup}
\textbf{Polgrupa} je par množice $\mathcal{S}$ skupaj z \textbf{asociativno binarno operacijo}. Pišemo: $(\mathcal{S}, \circ)$
\end{definition}

\begin{remark}
Kadar je jasno o kateri operaciji govorimo, pogosto govorimo kar o polgrupi $\mathcal{S}$
\end{remark}

\begin{example}
\begin{example_case}
$(\mathbb{N}, +), \underbrace{(\mathbb{Z}, +), (\mathbb{Q}, +), (\mathbb{R}, +), (\mathbb{C}, +), (\mathbb{N}, *),\dots}_{Niso\ samo\ polgrupe\ ampak\ kar\ grupe}$
\end{example_case}
\end{example}
\\
\\
Naj bo $(\mathcal{S}, \circ)$ polgrupa, po zakonu \ref{eq:asociative_law} o asociativnosti velja:
$$\forall x,y,z \in \mathcal{S}.\ (x \circ y) \circ z = x \circ (y \circ z)$$zato lahko okepaje spuščamo in vse to pišemo kot $x \circ y \circ z$. 
Kaj pa če imamo več kot tri elemente. Ali velja tudi:\\
$(x_1 \circ x_2) \circ (x_3 \circ x_4) = ((x_1 \circ x_2) \circ x_3) \circ x_4 = x_1 \circ (x_2 (\circ x_3 \circ x_4)) = \dots$





\end{document}