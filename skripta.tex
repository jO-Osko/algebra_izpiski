\documentclass[a4paper]{article}
%AMDG
\usepackage{amsmath, amsthm, amssymb}
\usepackage{hyperref}
\usepackage[slovene]{babel}
\usepackage[utf8]{inputenc}
\usepackage[T1]{fontenc}
\usepackage{epigraph}
\usepackage{pdftexcmds}
\usepackage{fancyref, nameref}
\usepackage{epigraph}
\usepackage{cleveref}
\usepackage{verbatim}

\parindent=0pt


% \epigraphsize{\small}% Default
\setlength\epigraphwidth{8cm}
\setlength\epigraphrule{0pt}

\usepackage{etoolbox}

\makeatletter
\patchcmd{\epigraph}{\@epitext{#1}}{\itshape\@epitext{#1}}{}{}
\makeatother


\newcounter{environment:definition_counter}

\newenvironment{definition}[1][\unskip]
{\vspace{0.5cm}\refstepcounter{environment:definition_counter}\textbf{Definicija \arabic{environment:definition_counter}: \textbf{#1}}\itshape}
{\bigskip}

\newcounter{environment:theorem_counter}

\newenvironment{theorem}[1][\unskip]
{\refstepcounter{environment:theorem_counter}\textbf{Izrek \arabic{environment:theorem_counter}:\textit{#1}} \qquad}
{\bigskip}

\newcounter{environment:statement_counter}

\newenvironment{statement}[1][\unskip]
{\refstepcounter{environment:statement_counter}\textbf{Trditev \arabic{environment:statement_counter}:\textit{#1}}}
{\bigskip}

\newcounter{example:example_counter}

\newenvironment{example}
{\textbf{Primer:}\\}
{\setcounter{example:example_counter}{0}}

\newenvironment{example_case}
{\refstepcounter{example:example_counter} \arabic{example:example_counter}.}
{\\}

\newenvironment{remark}
{\textbf{Opomba:}}
{}

\newenvironment{corollary}
{\underline{\textbf{Posledica:}}}
{}


\begin{document}
\title{Skripta za algebro2}
\author{Filip Koprivec}
\date{\today}
\maketitle

\epigraph{“If I find in myself desires which nothing in this world can satisfy, the only logical explanation is that I was made for another world.”}{--- \textup{C. S. Lewis}}
\newpage

\tableofcontents

\newpage

\begin{comment}
Start of text
\end{comment}

\section{Osnovne algebrske strukture}
\subsection{Binarne operacije}
\begin{definition}[Binarna Operacija]
\label{def:binary_operation}
(tudi dvočlena operacija) $\circ$ na množici $\mathcal{S}$ je preslikava iz $\mathcal{S} \times \mathcal{S} \  v \ \mathcal{S}$.\\ Torej $\circ : \mathcal{S} \times \mathcal{S} \ \to \ \mathcal{S}$
\end{definition}

\begin{example}
Osnovna zgleda binranih operacij na $\mathbb{Z}$ sta:

\begin{example_case}
Seštevanje: $(n,m) \mapsto n+m$
\end{example_case}
\begin{example_case}
Množenje: $(n,m) \mapsto n \times m$
\end{example_case}

\end{example}

Skalarni produkt v $\mathbb{R}^{2}$ \textbf{ni} binarna operacija.

Vektorski produkt v $\mathbb{R}^{3}$ \textbf{je} binarna operacija.
\\\\
\begin{definition}
\label{def:asociative_operation}
Operacija $\circ$ je \textbf{asociativna}, če ustreza enačbi
\begin{equation}
\label{eq:asociative_law}
\forall x,y,z \in \mathcal{S}.\ (x \circ y) \circ z = x \circ (y \circ z)
\end{equation}

\end{definition}

Enakost \ref{eq:asociative_law} imenujemo \textbf{Zakon o asociativnosti}\\
Operacije, ki jih bomo obravnavali bodo praviloma asociativne.

\begin{definition}
\label{def:comutative_operation}
Elementa $x,y \in \mathcal{S}$ \textbf{komutirata}, če velja 
\begin{equation}
\label{eq:comutative_law}
\forall x,y \in \mathcal{S}. x \circ y = y \circ x
\end{equation}
Enakost \ref{eq:comutative_law} imenujemo \textbf{Zakon o komutativnosti}

\end{definition}
\begin{remark}
Kadar je iz konteksta razvidno, o kateri operaciji govorimo, pogosto namesto "$\circ$ je komutativna rečemo tudi $\mathcal{S}$ je komutativna"
\end{remark}

\begin{example}
\begin{example_case}
Operacija + na $\mathbb{Z}$ je tako asociatitivna in komutativna
\end{example_case}
\begin{example_case}
Operacija * na $\mathbb{Z}$ je tako asociatitivna in komutativna
\end{example_case}
\begin{example_case}
Operacija - na $\mathbb{Z}$ \textbf{ni} niti asociativna niti komutativna
\end{example_case}
\begin{remark}
Na opracijo odštevanja gledamo kot na izpeljano operacijo in ne kot na samostojna operacijo, saj jo vpeljemo preko seštevanja in pojma nasprotnega elementa.
\end{remark}

\begin{example_case}
Naj bo $\mathcal{X}$ poljubna neprazna množica. Z $F(\mathcal{X})$ označimo množico vseh preslikav iz $\mathcal{X}$ v $\mathcal{X}$. Naj bosta $f, g \in \mathcal{X}$, potem je $(f,g) \mapsto f \circ g$ (kompozitum funkcij) binarna operacija na $F(\mathcal{X})$.

\begin{remark}
Operacija je asociativna, in kadar $|\mathcal{X}| \geq 2$ ni komutativna
\end{remark}
\end{example_case}
\end{example}

\begin{definition}
\label{def:identity_element}
Naj bo $\circ$ binarna operacija na na $\mathcal{S}$ in $e \in \mathcal{S}$. $e$ se imenuje \textbf{nevtralni element}, če velja 
\begin{equation}
\label{eq:identity_element}
\forall x \in \mathcal{S}. e \circ x = x \circ e = x
\end{equation}

\end{definition}

\begin{example}
\begin{example_case}
$0$ je nevtralni element za seštevanje na $\mathbb{Z}$.
\end{example_case}
\begin{example_case}
$1$ je nevtralni element za množenje na $\mathbb{Z}$.
\end{example_case}
\begin{example_case}
$id_{x}$ (identična preslikava) je nevtralni element za $F(\mathcal{X})$
\end{example_case}
\end{example}

\begin{remark}
Nevtralni element nima zagotovljenega obstoja (recimo $+$ na $\mathbb{N}$ ali $*$ na sodih celih številih).
\end{remark}

\begin{statement}
\label{st:identity_unique}
Če nevtralni element obstaja je en sam
\begin{proof}
\label{pr:identity_unique}
Naj bosta $f,e \in \mathcal{S}$ nevtralna elementa.
$$e = e \circ f \text{\ \ // Ker je f nevtralni element}$$
$$e \circ f = f \text{\ \ // Ker je e nevtralni element}$$
$$e = f $$
\end{proof}
\end{statement}

\begin{definition}
\label{def:left_identity}
Element $e'$ je \textbf{levi nevtralni element}, če velja:
\begin{equation}
\label{eq:left_identity_element}
\forall x \in \mathcal{S}. e' \circ x = x
\end{equation}
\end{definition}

\begin{definition}
\label{def:right_identity}
Element $e''$ je \textbf{desni nevtralni element}, če velja:
\begin{equation}
\label{eq:right_identity_element}
\forall x \in \mathcal{S}. x \circ e'' = x
\end{equation}
\end{definition}

\begin{remark}
Levih in desnih nevtralnih elementov je lahko več

\begin{example}
\begin{example_case}
$\circ : (x,y) \mapsto y$.

Vsak element je levi nevtralni element
\end{example_case}
\begin{example_case}
$0$ je desni nevtralni element za odštevanje v $\mathbb{Z}$
\end{example_case}
\end{example}
\end{remark}

\begin{statement}
\label{st:identity_left_right_both}
Naj bo za operacijo $\circ$ $e'$ levi nevtralni element, $e''$ pa desni nevtralni element. Tedaj velja $e' = e'' = e$(Sta si levi in desni nevtralni element enaka in je(sta) nevtralni element)
\begin{proof}
\label{pr:identity_left_right_both}
$$e' = e' \circ e'' = e''$$
\end{proof}
\end{statement}

\begin{definition}
\label{def:inner_operation}
Naj bo $\circ $ operacija na $\mathcal{S}$ in naj bo $\mathcal{T} \subseteq \mathcal{S}$. Rečemo, da je $\circ$ \textbf{notranja operacija na $\mathcal{T}$} ali da je množica \textbf{$\mathcal{T}$ zaprta za $\circ$ na $\mathcal{T}$} , če velja
\begin{equation}
\label{eq:inner_operation}
\forall t, t' \in \mathcal{T}. t \circ t' \in \mathcal{T}
\end{equation}

\end{definition}

\begin{example}
Množica $\mathbb{N}$ je zaprta za operaciji $+$ in $*$, ni pa zaprta za operacijo $-$.
\end{example}

\begin{definition}
\label{def:outer_operation}
Preslikavi iz $\mathcal{K} \times \mathcal{S}$ v $\mathcal{S}$ kjer $\mathcal{K} != \mathcal{S}$ rečemo \textbf{Zunanja binarna operacija}
\end{definition}

\begin{example}
\begin{example_case}
Množenje vektorja s skalarjem\\
$(\lambda, \vec{x}) \mapsto \lambda\vec{x}$, kjer je $(K = \mathbb{R}, S = \mathbb{R}^n)$\\
$\lambda (x_1, x_2, \dots , x_n) = (\lambda x_1, \lambda x_2, \dots ,\lambda x_n)$
\end{example_case}
\end{example}

\subsection{Polgrupe in monoidi}

\begin{definition}
\label{def:algebraic_structure}
\textbf{Algebrska struktura} je množica, opremljena z eno ali več operacijami (notranjimi ali zunanjimi), ki im ajo določene lastnosti
\end{definition}

\begin{definition}
\label{def:semigroup}
\textbf{Polgrupa} je par množice $\mathcal{S}$ skupaj z \textbf{asociativno binarno operacijo}. Pišemo: $(\mathcal{S}, \circ)$
\end{definition}

\begin{remark}
Kadar je jasno o kateri operaciji govorimo, pogosto govorimo kar o polgrupi $\mathcal{S}$
\end{remark}

\begin{example}
\begin{example_case}
$(\mathbb{N}, +), \underbrace{(\mathbb{Z}, +), (\mathbb{Q}, +), (\mathbb{R}, +), (\mathbb{C}, +), (\mathbb{N}, *),\dots}_{Niso\ samo\ polgrupe\ ampak\ kar\ grupe}$
\end{example_case}
\end{example}
\\
\\
Naj bo $(\mathcal{S}, \circ)$ polgrupa, po zakonu \ref{eq:asociative_law} o asociativnosti velja:
$$\forall x,y,z \in \mathcal{S}.\ (x \circ y) \circ z = x \circ (y \circ z)$$zato lahko okepaje spuščamo in vse to pišemo kot $x \circ y \circ z$. 
Kaj pa če imamo več kot tri elemente. Ali velja tudi:\\
$(x_1 \circ x_2) \circ (x_3 \circ x_4) = ((x_1 \circ x_2) \circ x_3) \circ x_4 = x_1 \circ (x_2 (\circ x_3 \circ x_4)) = \dots$

\begin{statement}
\label{def:semigroup_asociativity_operation}
Naj bo $(\mathcal{S}, \circ)$ polgrupa, $n \in \mathbb{N}$ in naj bo $x_1, x_2,\dots,x_n \in \mathcal{S}$. Tedaj je za vsak $n$ enakost izpolnjena na glede na postavitev oklepajev (izraz ima smisel, tudi kadar ne pišemo oklepajev).\\
$x_1 \circ x_2 \circ \dots \circ x_n = (\dots(x_1 \circ x_2) \circ \dots \circ x_n) = x_1 \circ (x_2 (\circ \dots \circ x_n)\dots) = \dots$
\end{statement}

\begin{proof}
\label{pr:semigroup_asociativity_operation}
Zgolj skica dokaza\\
Definirajmo:
$x:= x_1 \circ (x_2 (\circ \dots \circ x_n)\dots)$ in \\
$y:= $ naj bo kombinacija elementov $x_1 \dots x_n$, z drugače postavljenimi oklepaji\\
Indukcija na $n$:\\
$n \leq 3$: Očitno\\
Ker $n \leq 2$ velja $y = \underbrace{(u)}_{x_1, \dots, x_k} \circ \underbrace{(v)}_{x_{k+1}, \dots, x_n}$ Iz $k < n$ sledi:\\ $ y = (x_1 \circ w) \circ v \underbrace{=}_{Asociativnost (\ref{eq:asociative_law})} x_1 \circ ( w \circ v) $\\
Po I.P. ($w \circ v$ ima $n-1$ elementov):
$x = x_1 \circ (x_2 \circ \dots \circ x_{n-1})$
\end{proof}
Zato lahko oklepaje izpuščamo in pišemo kar:
$x_1 \circ x_2 \circ \dots \circ x_4$

\begin{definition}
\label{def:power_operation}
\textbf{Potenca elementa $x$.} Naj bo  $n \in \mathbb{N} - \{0\}$ in $x \in \mathcal{S}$
\begin{equation}
\label{eq:power_operation_natural}
x^n := \underbrace{x \circ x \circ \dots \circ x}_{n \text{elementov}}
\end{equation} 
\end{definition}

\begin{remark}
Brez asociativnosti ni definirano niti $x^3$
\end{remark}

\begin{remark}\\
Očitno velja:\\
$\forall n, m \in \mathbb{N}. x^n \circ x^m = x^{n+m}$ in \\
$\forall n, m \in \mathbb{N}. (x^n)^m = x^{n m}$
\end{remark}

\begin{definition}
\label{def:monoid}
\textbf{Polgrupa} z \textbf{nevtralnim elementom} se imenuje \textbf{monoid}.
\end{definition}
\begin{example}
\begin{example_case}
$(\mathbb{N}, +)$ \textbf{ni} monoid, $(\mathbb{N} \cup \{0\}, +)$ pa je.
\end{example_case}
\begin{example_case}
$(\mathbb{N}, *)$ je monoid
\end{example_case}
\begin{example_case}
$(F(\mathcal{X}), \circ)$ je monoid, nevtralni element je $id_{\mathcal{X}}$
\end{example_case}
\end{example}

\begin{definition}
\label{def:left_inverse}
Naj bo $( \mathcal{S}, \circ )$ monoid z nevtralnim elementom $e$. Element $y$ je \textbf{levi inverz}  elementa $x$, če velja: $y \circ x = e$.
\end{definition}

\begin{definition}
\label{def:right_inverse}
Naj bo $(\mathcal{S}, \circ)$ monoid z nevtralnim elementom $e$. Element $y$ je \textbf{desni inverz}  elementa $x$, če velja: $x \circ y = e$. 
\end{definition}

\begin{remark}
Levi in desni inverz nimata zagotovljenega obstoja, če pa obstajata ni nujno, da sta enolično določena.
\end{remark}

\begin{example}
\begin{example_case}
$f \in F(\mathcal{X})$ ima levi inverz $\iff f$ je injektivna\\
Če $f$ ni surjektivna ima lahko več levih inverzov, ki so izven $\mathcal{Z}_f$ lahko poljubno definirani.
\end{example_case}
\begin{example_case}
$f \in F(\mathcal{X})$ ima desni inverz $\iff f$ je surjektivna
\end{example_case}
\begin{example_case}
$f \in F(\mathcal{X})$ ima levi in desni inverz $\iff f$ je bijektivna
\end{example_case}
\end{example}

\begin{definition}
\label{def:inverse_element}
Element $y $ iz monida $\mathcal{S}$ je inverz elementa $x$ Če velja:
\begin{equation}
x \circ y = y \circ x = e
\end{equation}
Elementu, ki ima inverz rečemo da je \textbf{obrnljiv} in njegov inverz označimo z $x^{-1}$(To ni čisto korektno, saj bomo šele malo naprej pokazali, da ima vsak element en sam inverz). In tako dobimo 
\begin{equation}
\label{eq:inverse_element}
x \circ x^{-1} = x^{-1} \circ x = e
\end{equation}
\end{definition}

\begin{remark}
Če je operacija $\circ$ komutativna potem levi inverz, desni inverz in inverz za posamezen element sovpadajo
\end{remark}
\\
\\
\begin{statement}
\label{st:inverse_left_right_both}
Naj bo $(\mathcal{S}, \circ)$ monoid, Če je $y$ levi inverz elementa $x$ in je $z$ njegov desni inverz, potem $z=y=x^{-1}$
\end{statement}
\begin{proof}
$y = y \circ e = y \circ (x \circ z) = (y \circ x) \circ z = e \circ z = z$
\end{proof}
\begin{corollary}
Obrnljiv element monoida ima natanko en inverz.
\end{corollary}

\begin{corollary}
Če je $x$ obrnljiv element monoida $\mathcal{S}$ potem iz $y \circ x = e$ sledi $x \circ y = e$.
\end{corollary}
\\
\begin{statement}
\label{st:pairwise_inverse}
Če sta $x$ in $y$ obrnljiva, potem je obrnljiv tudi element $(x \circ y)$ in je njegov inverz $y^{-1} \circ x^{_1}$
\end{statement}
\begin{proof}
\label{pr:pairwise_inverse}
To je desni inverz:\\
$(x \circ y)  \circ (y^{-1} \circ x^{-1}) = x \circ (y  \circ y^{-1}) \circ x^{-1} = x \circ e \circ x^{-1} = x \circ x^{-1} = e$\\
in tudi levi inverz:\\
$(y^{-1} \circ x^{-1}) \circ (x \circ y) = y^{-1} \circ (x^{-1} \circ x) \circ y = y^{-1} \circ e \circ y = y{-1} \circ y = e$
\end{proof}
\begin{remark}
\label{rem:n_wise_inverse}
Seveda velja za $n$ elementov
\begin{equation}
\label{eq:n_wise_inverse}
(x_1 \circ  x_2 \circ \dots \circ x_n)^{-1} = x_{n}^{-1} \circ \dots \circ x_{2}^{-1} \circ x_{1}^{-1}
\end{equation}
\end{remark}

\begin{example}
\begin{example_case}
$(\mathbb{N} \cup \{0\}, +)$: edini obrnljiv element je 0.
\end{example_case}
\begin{example_case}
$(\mathbb{N}, *)$: edini obrnljiv element je 1
\end{example_case}
\begin{example_case}
$(\mathbb{Z}, *)$: edina obrnljiva elementa sta 1 in -1 
\end{example_case}
\begin{example_case}
$(\mathbb{Q}, *)$: Obrnljivi so vsi element razen 0
\end{example_case}
\begin{example_case}
$(F(\mathcal{X}), \circ)$: obrnljive so vse bijektivne preslikave
\end{example_case}
\end{example}
\begin{remark}
Poseben primer zadnje formule kadar je $x$ obrnljiv je tudi:$(x^n)^{-1} = (x^{-1})^n$ za $n \in \mathbb{N}$
\end{remark}

\begin{definition}
\label{def:integer_power}
\begin{equation}
\label{eq:integer_power}
n \in \mathbb{N}. x^{-n} := (x^n)^{-1} = (x^{-1})^n
\end{equation}
\end{definition}

\begin{definition}
\label{def:null_power}
\begin{equation}
\label{eq:null_power}
x^0 := e
\end{equation}

\end{definition}

Tako kadar je $x$ \textbf{obrnljiv} veljata enačbi

\begin{equation}
\label{eq:sum_of_powers}
\forall n, m \in \mathbb{Z}. x^n \circ x^m = x^{n+m}
\end{equation}

\begin{equation}
\label{eq:product_of_powers}
\forall n, m \in \mathbb{Z}. (x^n)^m = x^{n m}
\end{equation}

\begin{statement}
\label{st:reduction_rule}
Če je $x$ obrnljiv element monida $\mathcal{S}$ potm velja:
\begin{equation}
\label{eq:reduction_rule}
x \circ y = x \circ z \implies y = z
\end{equation}
In tudi
\begin{equation}
\label{eq:reduction_rule2}
y \circ x = z \circ x \implies y = z
\end{equation}
\begin{proof}
$$x \circ y = x \circ z \implies x^{-1} \circ x \circ y = x^{-1} \circ x \circ z \implies y = z$$
Druga enačba podobno
\end{proof}
\end{statement}

\begin{remark}
Iz enačbe $x \circ y = z \circ x$ v splošnem \textbf{ne} sledi $y = z$
\end{remark}

\subsection{Grupe}
\begin{definition}
\label{def:group}
\textbf{Monoid} v katerem je \textbf{vsak element obrnljiv}, se imenuje \textbf{grupa}. Grupa, v kateri je vsaka operacija komutativna se umenuje \textbf{komutativna grupa} ali \textbf{Abelova grupa}.
\end{definition}

Grupe delim na komutativne in nekomutativne(glede na lastnosti operacije) ter na končne in neskončne(glede na število elementov).

\begin{example}
\begin{example_case}
$(\mathbb{Z}, +)$, $(\mathbb{Q}, +)$,$(\mathbb{R}, +)$,$(\mathbb{C}, +)$
\end{example_case}
\begin{example_case}
$(\mathbb{N} \cup \{0\}, +)$ \textbf{ni} grupa
\end{example_case}
\begin{example_case}
$(\mathbb{R}, *)$: \textbf{ni} grupa, ker 0 ni obrnljiv
\end{example_case}
\end{example}

\begin{remark}
Vsak monoid 'skriva' grupo.
\end{remark}

\begin{definition}
\label{def:monoid_invertible_elements}
S \textbf{$\mathcal{S}^*$} označujemo množico vseh obrnljivih elementov monoida $\mathcal{S}$.
\end{definition}

\begin{statement}
\label{st:subgroup_of_monoid}
Če je $\mathcal{S}$ monoid je $\mathcal{S}$ grupa.
\end{statement}

\begin{proof}
\label{pr:subgroup_of_monid}
$x, y \in \mathcal{S}^* \implies x \circ y \in \mathcal{S}^*$ // Obrnljiv je tudi njun produkt, torej je množica je zaprta za $*$ \\
Ker je $*$ asociativen na $\mathcal{S}$ je asociativen tudi na $\mathcal{S}^*$\\
$e \in \mathcal{S}^*$ saj je enota inverz sami sebi\\
$x \in \mathcal{S}^* \implies x^{-1} \in \mathcal{S}^*$ // Inverz inverza je kar element sam
\end{proof}

\begin{example}
\begin{example_case}
$(\mathbb{N} \cup \{0\}, +)$: $(\mathbb{N} \cup \{0\}, +)^* = {0}$
\end{example_case}
\begin{example_case}
$(\mathbb{Z}, +)$: $(\mathbb{Z}, +)^* = {-1,1}$ 
\end{example_case}
\begin{example_case}
$(\mathbb{Q}, *)$: $(\mathbb{Q}, *)^* = \mathbb{Q} - {0}$
\end{example_case}
\begin{remark}
Grupam z enim elementom pravimo \textbf{trivialne} grupe.
\end{remark}

\begin{example_case}
$(F(\mathcal{X}), \circ)$: $(F(\mathcal{X}), \circ)^* = \{f: \mathcal{X} \to \mathcal{X} | f \ \text{je bijekcija}\}$
\end{example_case}
\end{example}

\begin{definition}
\label{def:symetric_group}
Množico $Sim(\mathcal{X})$ imenujemo \textbf{simetrična grupa} (množice $\mathcal{X}$). 
\begin{equation}
\label{eq:general_symetric_group}
Sim(\mathcal{X}) := \{f: \mathcal{X} \to \mathcal{X} | f \ \text{je bijekcija}\}
\end{equation}
Njene emelente(bijektiven preslikave iz $\mathcal{X}$ v $\mathcal{X}$ pa imenujemo \textbf{permutacaije} (množice $\mathcal{X}$).
\end{definition}

\begin{remark}
Če je množica končna jo praviloma označimo z $\{1,2,\dots, n\}$, njej pripadajočo grupo permutacij pa z
\begin{equation}
\label{eq:finite_symetric_group}
\mathcal{S}_n := Sim(\{1,2,\dots,n\})
\end{equation}
\end{remark}





























\end{document}